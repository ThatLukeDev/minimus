\documentclass{article}

\usepackage{caption}
\usepackage[newfloat]{minted}
\captionsetup[listing]{position=top}

\title{OS Development}
\author{Luke}

\begin{document}

\maketitle

\begin{abstract}

This document serves as a reference for the Minimus operating system.

\end{abstract}

\newpage

\tableofcontents

\newpage

\section{Bootloader}

\subsection{Headers}

The disk is ordered into sectors, of size 200$_{16}$, starting from sector 1\cite{sector size}.
When the BIOS loads the OS, it copies the first sector,
which would be the first 512 bytes, from the disk into memory,
starting at 7C00$_{16}$\cite{7c00}.

Memory is also ordered into segments, of size 10$_{16}$\cite{memory segments}. This means that memory
addresses can overlap, for example:
The address 0000:7C00$_{16}$ is the same as 0700:0C00$_{16}$.

You must ensure that the bytes at 1FE$_{16}$ and 1FF$_{16}$
contain values 55$_{16}$ and AA$_{16}$ respectively\cite{55aa}.
This is called the magic number, and is used to differentiate
bootable disks from non-bootable disks.

When the BIOS hands control to the OS, the CPU will be in 16 bit (real) mode\cite{16bitstart},
which means it is using 16 bits per instruction. This is because the CPU is in
16 bit mode by default. You must ensure that your program code for the bootloader
begins in 16 bit (assuming real) mode.

\begin{listing}[h]
\caption{Booatloader}
\begin{minted}[linenos,frame=single]{nasm}
[org 0x7c00]		; memory load location
[bits 16]		; real mode
\end{minted}
\end{listing}

\newpage

\begin{thebibliography}{99}
	\bibitem{7c00}
		Compaq Computer Corporation, Phoenix Technologies Ltd, Intel Corporation (1996)
		\emph{BIOS Boot Specification}
		pg 29 ch 6.5.1
	\bibitem{int 13h}
		Ralf Brown (2000)
		\emph{Ralf Browns Interrupt List}
		INT 13
	\bibitem{sector size}
		IDEMA (2013)
		\emph{The Advent of Advanced Format}
		pg 1
	\bibitem{memory segments}
		Intel Corporation (2016)
		\emph{Intel 64 and IA-32 Architectures Software Developer’s Manual}
		vol 1 pg 1.6 ch 1.3.4
	\bibitem{55aa}
		Compaq Computer Corporation, Phoenix Technologies Ltd, Intel Corporation (1996)
		\emph{BIOS Boot Specification}
		pg 12 ch 3.3
	\bibitem{16bitstart}
		Intel Corporation (2016)
		\emph{Intel 64 and IA-32 Architectures Software Developer’s Manual}
		vol 3a pg 11.13 ch 1.9.1
\end{thebibliography}

\end{document}
