\documentclass{article}

\title{OS Development}
\author{Luke}

\begin{document}

\maketitle

\begin{abstract}

This document serves as a reference for the Minimus operating system.

\end{abstract}

\newpage

\tableofcontents

\newpage

\section{Bootloader}

The disk is ordered into sectors, of size 200$_{16}$, starting from sector 1.
When the BIOS loads the OS, it copies the first sector,
which would be the first 512 bytes, from the disk into memory,
starting at 7C00$_{16}$.

Memory is also ordered into segments, of size 10$_{16}$. This means that memory
addresses can overlap, for example:
The address 0000:7C00$_{16}$ is the same as 0700:0C00$_{16}$.

You must ensure that the bytes at 1FE$_{16}$ and 1FF$_{16}$
contain values 55$_{16}$ and AA$_{16}$ respectively.
This is called the magic number, and is used to differentiate
bootable disks from non-bootable disks.

When the BIOS hands control to the OS, the CPU will be in real mode,
which means it is using 16 bits per instruction. You must
ensure that your program code for the bootloader begins in real mode.

\newpage

\begin{thebibliography}{99}
\end{thebibliography}

\end{document}
